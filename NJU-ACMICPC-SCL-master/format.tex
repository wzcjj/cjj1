\usepackage{amssymb}
\usepackage{amsthm}
\usepackage{fontspec,xunicode,xltxtra}
\usepackage{titlesec}
\usepackage{indentfirst}
\usepackage[BoldFont,SlantFont]{xeCJK}
\usepackage{fancyhdr}
\usepackage{graphicx}
\usepackage{listings}
\usepackage{printlen}
\usepackage{ifthen}
\usepackage[savepos]{zref}
\usepackage{multicol}
\usepackage{sectsty}
% \XeTeXinputencoding "cp936"

\setCJKfamilyfont{kai}{SimSun} % {KaiTi_GB2312}
\setCJKfamilyfont{hei}{SimSun} % {STXihei}
\setCJKmainfont{SimSun}
\newfontfamily{\monotype}{Courier New}

\pagestyle{fancy}
\rhead{{\sf\thepage}}
\lhead{\kai ACM/ICPC Code Library}

\fancyhead[L]{\sf\kai \leftmark} 
\fancyhead[R]{\sf\kai \rightmark}



\newcommand{\kai}{\CJKfamily{kai}}
\newcommand{\hei}{\CJKfamily{hei}}

\setlength{\parindent}{2.2em}

\renewcommand{\contentsname}{\hei 目录}


% settings for listings
\lstset {
  basicstyle = \small\monotype,
  language = C++,
  tabsize = 2,
  breaklines = true,
  breakindent = 1.1em,
  numbers=right,
  stringstyle=\monotype,
  numberstyle=\footnotesize\ttfamily,
  firstnumber=last,
  basewidth={0.55em, 0.5em}
}

% font of section header
\allsectionsfont{\hei}

% an amazing script
% converts an line-number to arbitrary string
\let\othelstnumber=\thelstnumber
\def\createlinenumber#1#2{
    \edef\thelstnumber{%
        \unexpanded{%
            \ifnum#1=\value{lstnumber}\relax
             \tt #2%
            \else}%
        \expandafter\unexpanded\expandafter{\thelstnumber\othelstnumber\fi}%
    }
    \ifx\othelstnumber=\relax\else
      \let\othelstnumber\relax
    \fi
}

